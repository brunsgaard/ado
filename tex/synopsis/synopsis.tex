\documentclass[10pt,a4paper,final,oneside,openany,article]{memoir}

\input{prelude}

\title{Approximate distance oracles in practice}
\author{
  Jonas Brunsgaard -- \texttt{jonas.brunsgaard@gmail.com} -- \texttt{msn378} \\
  University of Copenhagen
}

\date{16th September 2013}

\bibliography{bibliography}

\begin{document}
\maketitle

\chapter{Project summary}
Let $G = (V, E)$ be an undirected weighted graph with $|V| = n$ and
$|E| = m$. Let $k > 1$ be an integer. Mikkel Thorup and Uri Zwick has
presented approximate distance oracles\cite{tu}, a data structure of size
$O(kn^{1+1/k})$ that can be preprocessed in $O(kmn^{1/k})$ expected time and
have the ability to answer subsequent distance queries in $O(k)$ time.
The approximate distance returned from the preprocessed data structure, denoted
by $\hat{\delta}$, is of stretch at most $2k-1$, meaning $\delta(u,v)\leq
\hat{\delta}(u,v)\leq(2k-1)\delta(u,v)$ where $u$ and $v$ are vertives in $G$

The aim of this project is to show an implementation and run experiments over
the approximate distance oracles algorithm. The experiments will focus on
space consumption of the preprocessed data structure and the magnitude of the
actual stretch. Results and will be discussed an compared to the theoretical
limits of the algorithm. Experiments will be performed on real world data from
the Stanford Large Network Dataset Collection.

\chapter{Motivation}
The project will deliver a qualified guess regarding the expected value of the
stretch for graphs classes used in our experiments. This may be of interest to
specific industries such as e.g. telecommunication.
Furthermore the project will supply an open source reference implementation
that can be used in future experiments regarding approximate distance oracles.

\chapter{Problem statement}
How much does the stretch and space consumption differ from the theoretical
limits, when approximate distance oracles are applied to real world data sets?

\chapter{Scope}
Our experiments are limited to stretch and the size of the precomputed data
structure.

\chapter{Learning objectives}
\begin{itemize}
    \item Implement non-trivial graph algorithms.
    \item Explain approximate distance oracles.
    \item Prove properties of approximate distance oracles.
    \item Argue about the quality of approximate distance oracles implementations.
\end{itemize}

\chapter{Tasks to complete}
To finish this project the following tasks has to be completed. The week
number indicates that a given task is the main focus that week.

\begin{itemize}[leftmargin=2cm]
    \item[week 39] Work with the article to understand approximate distance
                   oracles.
    \item[week 39] Write an implementation.
    \item[week 40] Test implementation on smaller data sets and verify
                   correctness of results.
    \item[week 41] Write code that can determine the stretch and space
                   consumption of the produced data structure.
    \item[week 42] Improve implementation in regard to hashing, choice of SSSP
                   algorithm, etc.
    \item[week 43] Benchmark on graphs from Stanford Large Network Dataset
                   Collection.
    \item[week 45] Document the work performed in report.
\end{itemize}

\defbibheading{bibliography}{\chapter{Literature}}
\printbibliography

\end{document}
