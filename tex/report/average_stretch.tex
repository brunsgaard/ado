\chapter{Average stretch of estimates}
From \autoref{eq:stretchinterval} we know approximate distance oracles
produces results with a stretch guarantee. That is a worst-case multiplicative
factor increase of path length between a pair of vertices. But this worst case
guarantee may differ substantial from the average multiplicative factor on the
produces paths. For a single shortest path query the stretch of the result is
denoted $\hat{s}$ and is as the fraction of the exact shortest path.
\begin{equation}
    \hat{s}(u,v) = \frac{\hat{\delta}(u,v)}{\delta(u,v)}
\end{equation}
Thus there will also exists an average such that
\begin{equation}
    \bar{s} = \frac{\sum\limits_{(u,v) \in V \times V} \hat{s}(u,v) }{|V|^2}
\end{equation}

We will try to dertermine $\bar{s}$ through experiments over data from
Stanford Large Network Dataset Collection.

Graphs can have many properties and two graphs may not be suited for using the
algorithm. We have choosen two types of graphs where applications have the need
to answer fast.





\subsection*{Classes of graph we will test}
I have choosen some different areas of graph to see is the results differ.
Roadnet and internet topologies

\begin{description}
  \item[Internet topologies] \hfill
        \begin{description}
        \item[Skitter] \hfill \\
                Internet topology graph. From traceroutes run daily in 2005 -
                \url{http://www.caida.org/tools/measurement/skitter}. From several
                scattered sources to million destinations. 1.7 million nodes, 11
                million edges.

        \item[AS-733] \hfill \\
                The graph of routers comprising the Internet can be organized into
                sub-graphs called Autonomous Systems (AS). Each AS exchanges traffic flows
                with some neighbors (peers). We can construct a communication network of
                who-talks-to- whom from the BGP (Border Gateway Protocol) logs.
        \end{description}
  \item[Road networks] \hfill
        \begin{description}
        \item[California] \hfill \\
                A road network of California. Intersections and endpoints are represented
                by nodes and the roads connecting these intersections or road endpoints are
                represented by undirected edges.

        \item[Texas] \hfill \\
                A road network of Texas. Intersections and endpoints are represented
                by nodes and the roads connecting these intersections or road endpoints are
                represented by undirected edges.

        \item[Pennsylvania] \hfill \\
                A road network of Pennsylvania. Intersections and endpoints are represented
                by nodes and the roads connecting these intersections or road endpoints are
                represented by undirected edges.
    
        \end{description}
\end{description}

The theory of determining the average stretch from experiment is rather
trivial but to sample results data I implemented approximate distance oracles.
