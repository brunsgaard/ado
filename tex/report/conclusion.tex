\chapter{Conclusion}
\label{sec:conclusion}

Based on this report I can conclude that making an efficient implementation
of approximate distance oracles, capable of handling data sets with millions
of nodes, requires the use of heavily performance optimized language, memory
efficient data structures and concurrent programming, where the algorithm
allows for it.

In this report I have provided empirical data that suggests the average actual
stretch of $k=2$ approximate distance oracles data structures is considerably
lower than the worst case stretch of $3$. I have tested the actual stretch
on two internet topologies and three road network and the maximum value for
$\bar{s}$ I have been able to produce is $1.56$. The internet topologies
domain has a combined mean of $1.48$ and the road networks $1.07$, which I
find to be a fine overall result.

My experiments also suggest that the average actual stretch produced by
approximate distance oracles on graphs within the same domain is very similar.
Also the distribution of the graph stretches seems related for the results
generated within the domain.

For graphs in the domain of internet topologies, I have observed that both
data sets used in the experiments produce results in the same range and their
distribution of the produced stretches are similar. So I find that $\bar{s}
\approx 1.5$ is a reasonable indicator for the average actual stretch found in
internet topologies.

Furthermore, the experiments run over road networks for California, Texas and
Pennsylvania produce exciting results. Not only are the distributions and
averages of the actual stretches produced from the created data structures
more or less identical. The mean value is found to be approximately $1.1$,
which I find impressive. This indicates that graphs from this domain possess
similar properties that makes approximate distance oracles produce very good
estimates.
