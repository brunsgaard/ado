\begin{abstract}
Approximate distance oracle is a data structure presented by Thorup and
Zwick\cite{tu} that allows for approximate shortest path queries in
constant time on undirected connected graphs with positive edge weights.
The approximated distances is said to be of \emph{Stretch-3}, if the data
structure is constructed in such a way that the produced estimates, is at most
a factor three longer than the actual shortest path.

In this report I look at the actual stretches produces by \emph{Stretch-3}
approximate distance oracles. Through experiments I attempt to indicate
a value of the expected actual stretch, in graphs representing internet
topologies and road networks. To perform the experiments, a high performance
C++ implementation of approximate distance oracles has been developed and is
discussed in the report.

Based on experiments over internet topologies from Stanford Large Network
Data\cite{snapnets} collection. I find that the tested graphs on average produce stretches
of $\sim1.5$. When conducting the experiments over road networks from California,
Texas and Pennsylvania the average actual stretch is $\sim1.1$. I find the graphs
of each domain to produce very similar results and the distribution of the
actual stretches produced is also a like.

Thus I find these values to be reasonable indicators for the average actual
stretch for their domains.
\end{abstract}
